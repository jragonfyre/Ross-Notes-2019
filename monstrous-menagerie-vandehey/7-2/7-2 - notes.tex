\documentclass{article}

\title{Monstrous Menagerie with Vandehey 7/2}
\author{Jason Schuchardt}

\usepackage{jragonfyre}
\newcommand\cft[1]{\frac{1}{#1+}}
\newcommand\hoint{[0,1)}
\newcommand\recip[1]{\frac{1}{#1}}
\newcommand\cylset[1]{C_{[{#1}]}}
\newcommand\diam{\operatorname{diam}}

\theoremstyle{remark}
\newtheorem{exercise}{Exercise}

\begin{document}
\maketitle

\section{Probability and Invariant measures.}

\begin{definition}
    $\mu$ is a \emph{probability measure} if $\mu(X)=1$.
\end{definition}

If $\mu(X)$ is finite, we can always renormalize to get a
probability measure:
\[ \mu^*(A) = \frac{\mu(A)}{\mu(X)}. \]

Sometimes $\mu(X)$ is infinite. These are basically the only two
possibilities (other than the zero measure, which is boring).

\begin{definition}
Given $(X,\calA,T,\mu)$ 
(space, $\sigma$-algebra, transformation, measure)
we say $\mu$ is \emph{$T$-invariant,} if 
for all $A\in \calA$, 
\[\mu(T\inv A)=\mu(A),\]
where $T\inv(A) = \set{x\in X : Tx\in A}$.
\end{definition}

Two questions here: Why is it important? Because almost 
everything we want to do requires invariance.

If $\mu$ is not invariant,
define $\mu_k(A) = \mu(T^{-k}A)$,
and then we can take a sort of limit of $\mu_k$ to get an
invariant measure.

The other question is: Why is it $T\inv A$, why not just
$TA$? Because $T\inv$ preserves all of the information.
I can start with two points $x$ and $y$ and apply $T$ and
get a single point. For example with base $b$ expansions,
two points which differ in their first digit end up at the 
same point after applying $T$.

On the other hand, for $T\inv$ we know where we came from,
we can just apply $T$ to any point in $T\inv A$.
$TT\inv x = x$, but $T\inv T x = ?$.

\subsection{Proving invariance}

\begin{theorem}
    Suppose $\AAA$ is a semi-algebra that generates a $\sigma$-
    algebra $\calA$. If $\mu(T\inv A) = \mu(A)$ for all
    $A\in\AAA$, then $\mu$ is $T$-invariant.
\end{theorem}

\begin{proof}
    Notetaker's proof sketch:

    Define $\mu'(A) = \mu(T\inv A)$. $\mu'$ is also a measure
    on $(X,\calA)$, so by the uniqueness of the Caratheodory
    Extension Theorem, and the fact that the given information
    is that $\mu'(A)=\mu(A)$ for $A\in\AAA$, we conclude 
    that $\mu=\mu'$ as desired.
    (Uniqueness requires $\sigma$-finiteness)
\end{proof}

\begin{example}
    For base $b$, $\lambda$ is $T$-invariant.
    \[ T\inv x = \set*{\frac{0}{b} + \frac{x}{b},
    \frac{1}{b} + \frac{x}{b}, \ldots, 
    \frac{b-1}{b} + \frac{x}{b}}\]
    \[ \lambda(T\inv [x,y]) = 
    \lambda \of*{\bigcup_{i=0}^{b-1} \braks*{\frac{i+x}{b}, \frac{i+y}{b}}
    }
    = \sum_{i=0}^{b-1} \lambda\of*{\braks*{\frac{i+x}{b},\frac{i+y}{b}}}
    = \sum_{i=0}^{b-1} \frac{y-x}{b} = y-x
    \]

    Since the semi-algebra of intervals generates the 
    Lebesgue $\sigma$-algebra, $\lambda$ is $T$-invariant.
\end{example}

\begin{exercise}
    Prove that $\lambda$ is not invariant for regular
    CF expansions.
\end{exercise}

\begin{definition}
    An \emph{algebra} is a collection $\AAA$ of subsets of $X$ satisfying
    \begin{enumerate}
        \item $\nullset \in \AAA$,
        \item if $A,B \in \AAA$, then $A\cap B\in \AAA$,
        \item if $A\in \AAA$, then $X\setminus A \in \AAA$.
    \end{enumerate}
\end{definition}

To get an algebra from a semi-algebra, take all finite disjoint unions of subsets
and add them to the semi-algebra. (This is also the smallest
algebra containing the semi-algebra).

\begin{definition}
    A monotone class is a collection $\CCC$ of subsets of $X$ 
    satisfying
    \begin{enumerate}
        \item If $E_1\subseteq E_2 \subseteq E_3\subseteq \cdots $
            are all in $\CCC$, then 
            \[ \bigcup_{i} E_i \in\CCC \]
        \item If $F_1\supseteq F_2\supseteq F_3 \supseteq \cdots $
            are all in $\CCC$, then 
            \[ \bigcap_i F_i \in \CCC \]
    \end{enumerate}

    Once again, if you start with any collection that you'd like,
    there is a smallest monotone class containing that collection.
\end{definition}

\begin{lemma}
    Let $\AAA$ be any algebra of $X$. Then the monotone class
    generated by $\AAA$ is the same as the $\sigma$-algebra
    generated by $\AAA$, $\sigma(\AAA)$.
\end{lemma}
\begin{proof} Omitted, see example below. You want to prove 
    that the monotone class is a $\sigma$-algebra and vice-versa.
\end{proof}

\begin{example}
    If $E_1\subseteq E_2 \subseteq E_3 \subseteq \cdots $
    are in the $\sigma$-algebra, then
    \[\bigcup_i E_i\]
    is in the $\sigma$-algebra too.

    For intersections, observe that we have 
    this helpful equality
    \[\bigcap_i F_i = X\setminus \bigcup_i (X\setminus F_i). \]
\end{example}

\begin{proof}[Proof of theorem]
    Let 
    \[\CCC = \set{A\in \calA : \mu(T\inv A) = \mu(A) }.\]

    We know by assumption, $\AAA \subseteq \CCC$, and 
    by construction $\CCC\subseteq \calA$.
    Moreover, if $\mu(T\inv A_i) = \mu(A_i)$, for some 
    collection of disjoint $A_i$s, then 
    it must hold for their union as well. 
    \[\mu\of*{T\inv \bigcup_i B_i}
    = \mu\of*{\bigcup_i T\inv B_i}
    = \sum_i \mu(T\inv B_i)
    = \sum_i \mu(B_i)
    = \mu\of*{\bigcup_i B_i}.
    \]

    Thus $\CCC$ contains the algebra generated by $\AAA$.
    Next we'd like to apply the lemma, by showing that $\CCC$
    is a monotone class.

    Let $E_1\subseteq E_2 \subseteq \cdots $ be in $\CCC$.
    Let $E = \bigcup_i E_i$.
    \begin{align*}
        \mu(T\inv E) &= \mu\of*{T\inv{\bigcup_i E_i}} \\
        &= \lim_{i\to \infty} \mu(T\inv E_i)\\
        &\text{ (maybe requires $\mu$ finite/$\sigma$-finite, certainly true when $\mu(X)=1$)} \\
        & = \lim_{i\to\infty} \mu(E_i) \\
        & = \mu\of*{\bigcup_i E_i} \\
        &= \mu(E).
    \end{align*}

    Similarly, we get the other property, so $\CCC$ is a
    monotone class.

    Recall $\CCC\subseteq \calA$. By lemma, $\calA$ is the smallest
    monotone class containing the algebra, but $\CCC$ is a 
    monotone class containing the algebra. Thus $\CCC = \calA$.

    Recalling that $\CCC$ is defined to be the collection 
    of all invariant sets, 
    we have invariance of the whole $\sigma$-algebra, so 
    we are done.
\end{proof}

\begin{example}
    Other invariant measures:

    Let 
    \[ \delta_x(A) = \begin{dcases} 1 & x\in A \\ 0 & x\not\in A \end{dcases}
    \]
    be the Dirac measure.

    For base-$2$,
    \[ \frac{1}{2} (\delta_{1/3} + \delta_{2/3}) \]
    is an invariant probability measure.
\end{example}

Story time: There was a mathematician, perhaps Serre, who would 
shout out in lectures, ``Your notation sucks!''

The graduate students appreciated this, so they decided
to make him a ``Your notation sucks!'' shirt, but they needed
to presernt it correctly.

So they designed a lecture with the worst possible notation,
the conjugate of capital xi, $\Xi$ divided by itself:
\[\frac{\overline{\Xi}}{\overline{\Xi}}\]
He said nothing.
So they had to present the shirt after the lecture.
(End of story time)

$\delta_{1/3}$ ($\delta_{2/3}$) only cares about $1/3$ ($2/3$).
\[ T\inv \frac{1}{3} = \set*{\frac{1}{6}, \frac{2}{3}}, \]
which still has measure $1/2$.

In general, we can construct invariant measures from any periodic
point.

\section{Why is invariance actually useful? - Poincar\'e Recurrence}
\begin{theorem}[Poincar\'e Recurrence]
    Let $(X,T,\mu)$ be a dynamical system, with $\mu$ a 
    $T$-invariant probability measure.

    Then for any set $A$ with $\mu(A) > 0$, almost all 
    points in $A$ return to $A$ infinitely often (as we 
    iterate $T$).
\end{theorem}

It's possible that when we apply $T$ to a point in $A$,
it might leave $A$ and never return, but this happens almost 
never. Almost every point will land in $A$ infinitely often.

Application: Consider base-$b$ expansion.
For every cylinder set $C_s$, $C_s$ has positive measure.
So for almost all points $x$ that start with $s$,
we see $s$ infinitely often in $x$.

This is maybe not surprising in base-$b$.

Is it true for $\beta$-expansions? Continued fractions?
L\"uroth series? Answer: yes, with the appropriate invariant 
measure.


\end{document}