\documentclass{article}

\usepackage{jragonfyre}
%\usepackage{mathscr}

\title{Fixed Points with Bergelson, 6/25}

\begin{document}

\maketitle

Assignment: Who was Norbert Wiener? He wrote \emph{Cybernetics} and coined the term.

Rule: Look up any name that is mentioned on Wikipedia and find out who they were and what 
math they might have done.

Norbert Wiener said that the secret to being a good mathematician is to not be afraid 
to ask questions. 

\section{What is a fixed point?}

\begin{definition}
If $f:X\to X$ is a mapping ($X$ is some set or space with no assumptions,
and there are no assumptions on $f$ either),
and $x\in X$ satisfies $f(x)=x$, then $x$ is called a \emph{fixed point}
\end{definition}

If you want to solve a differential equation, or establish the existence of Haar measure.
(Counselors should talk about measure.)

Modern mathematics is not the result of hard work, but hitting the problem with a clever fixed point
theorem. You define a map that moves measures to measures, and if there is a fixed point, then you
have your solution.

There is no theory of fixed points, just a collection of interesting theorems. For the same reason,
there is no theory of inequalities, just bunches of inequalities.

\begin{proposition}
    Claim: if \[ f:[0,1] \to [0,1] \]
    is a continuous function, then there exists
    $x$ such that $f(x)=x$.
\end{proposition}

Picture: Draw $y=x$ on the square $[0,1]\times [0,1]$ and the graph of $f$. 
They intersect in a fixed point.

Suppose you know this theorem. What are questions about this theorem?

Question: If it's an open interval, it fails, why?

Exercise: Find counterexamples: 
\begin{enumerate}[1.]
        \item $I$ is replaced by $[0,1)$.
        \item $I$ is replaced by $\RR$.
        \item $I$ is replaced by $\TT:=\RR/\ZZ$, the torus. Another way to think of it
          as the circle, $S^1:=\set{z\in\CC : |z| = 1}$. 
\end{enumerate}

There are continuous functions that are nowhere differentiable. Weierstrass gave an example in
an unreadable paper. The function was something like
\[ \sum_{n\inN} a^n \sin(b^n x) \]

In good mathematics, nowhere differentiable functions still show up,
like in Brownian motion. (Norbert Wiener) Brownian motion was
discovered by a guy, named Brown, who was looking in a microscope,
and observed pieces of pollen jumping around.

Fact: A typical continuous function is nowhere differentiable.

Question: What is typical? Good question, but I won't tell you,
because it's too early in the course.

Exercise: Look up Banach. We will do the Banach fixed point theorem
next time. Banach showed that a typical function is nowhere
differentiable.

Exercise: Know who Hardy is. Hardy wrote a nice paper about
Weierstrass functions. Characterized for which $a$ and $b$
the function is nowhere differentiable.

Question: What if you lose continuity? 

If you lose continuity you shouldn't expect anything.

Question: What if we generalize to a cube?

Back to continuity. 

Which functions that are not continuous are still nice?
(Jad) What if you require your function to be Lipschitz?
Good, but that's skipping ahead.

Monotone functions. (Should be a topic for the seminar.)

\begin{definition}
    If for all $x<y\in\RR$ $f(x) \le f(y)$, then
    $f$ is (monotone) non-decreasing.
\end{definition}

Exercise: Define non-increasing, strictly increasing, 
and strictly decreasing.

Discontinuities. There are discontinuities of two kinds.

Picture: Monotone function that jumps up. Left and right limits
exist. This is the first kind of discontinuity.

The other possibility is that the function is wild, and one of
the one sided limits doesn't exist.

Properties of monotone functions 
\[ f: \RR\to\RR \]
\begin{itemize}
    \item Monotone functions only have ``nice'' discontinuities.
    \item ``Theorem of Luke'': Monotone functions have at most
        countably many discontinuities.
    \item Monotone functions are almost everywhere
        differentiable.
\end{itemize}

Let $\powerset{\NN}$ be the set of all subsets of $\NN$,
which is also written as $2^\NN$.

Aside on the notation $2^\NN$. If $S=\set{1,2,\ldots,n}$, then $\powerset{S}$ has $2^n$
elements.

\begin{theorem}[Cantor]
$\powerset{\NN}$ is not countable.
\end{theorem}

Do theorem in seminar.

Now we'll prove the theorem.

\begin{theorem}
    Monotone functions have at most countably many discontinuities.
\end{theorem}

(Fatih) For each discontinuity we pick a distinct rational.

\begin{proof}
Picture: Graph a monotone function with jumps. Each jump creates
an interval on the $y$-axis.

In other words, for each discontinuity $x$, the left limit $L_x$
and right limit $R_x$ at $x$ exist and are different by the first
property. This gives us nonempty disjoint intervals $(L_x,R_x)$
on the $y$-axis,
each of which contains a rational number, since the rationals are
dense. Then for each $x$, we choose $q_x\in (L_x,R_x)\cap \QQ$.

This gives an injection from the discontinuities of $f$,
$\Disc(f)$ to $\QQ$, which is countable. Thus the discontinuities
of $f$ are at most countable.
\end{proof}

Fact: If $(I_\tau)_{\tau\in S}$ is a set of disjoint intervals
of nonzero length in $\RR$ then the collection is at most countable.

\begin{proof}
    For each $\tau\in S$, choose $q_\tau \in I_\tau \cap \QQ$,
    which is nonempty, since $I_\tau$ has nonzero length.
    This gives an injection from $S$ to $\QQ$, proving the 
    claim.
\end{proof}

\begin{definition}
    \emph{Almost everywhere} means that the complement of the 
    set has \emph{measure zero}. A set $S\subseteq \RR$ has 
    \emph{measure zero} if it can be covered by a possibly 
    infinite collection of intervals with arbitrarily small 
    total length.

    In other words, for every $\epsilon > 0$, there exist
    intervals $I_n$, $n\in\NN$ such that 
    \[ S\subseteq \bigcup_{n\inN} I_n,\]
    and 
    \[ \sum_{n\inN} |I_n| < \epsilon. \]
\end{definition}

(Tae Kyu) If a subset of $\RR$ has measure zero, does it have
to be at most countably infinite?

Sets of measure zero can be uncountable. 
Cantor sets should be covered in the seminar.

It's hard to intuitively measure the difference in cardinalities.
Both the rationals and irrationals are dense, but there are many
more irrationals than rationals in a precise sense. 

Is it clear that $\QQ$ has measure zero.
Choose a sequence $(r_n)$ containing all the rational numbers.
Cover $r_n$ with the interval 
\[ I_n = \of*{r_n-\frac{\epsilon}{2^{n+1}}, r_n + \frac{\epsilon}{2^{n+1}}}. \]
The total length of these intervals is $\epsilon$.

\section{The middle thirds Cantor set}

Picture: Iteratively remove the middle third from the intervals 
remaining at each step.

Let $K_n$ be the set at the $n$th set, and define the
Cantor set to be
\[ \mathscr{C} = \bigcap_{n=1}^\infty K_n. \]

The best way to prove that $\mathscr{C}$ is uncountable is to
realize that the Cantor set encodes an infinite tree of choices.

Exercise: The total length of the removed intervals in
the middle thirds Cantor set is $1$.

Picture: $f(x)=ax(1-x)$ graphed over the unit square. 
If $a=4$, then the vertex of the parabola is at $y=1$. 

This family of functions is called the logistic family, $a>0$.

When $a>4$, we can look at the points that leave the interval 
$[0,1]$. Then iterating the map, taking $f\circ f$, 
$f\circ f \circ f$, and so on, we can look at what points 
always stay in the interval $[0,1]$ and never leave.

Exercise: If $a>4$, then the set of points of $[0,1]$ that
stay in $[0,1]$ after infinitely many iterations is uncountable,
and has measure zero (hard exercise), 
and is homeomorphic to a Cantor set.

We have notions of smallness and largeness now. 

Seminar: Prove Cantor sets are nowhere dense.

Smallness is measure zero, and largeness is that the complement
is small. Nowhere dense is another notion of smallness.

Can we prove that monotone functions are almost everywhere 
differentiable.

If a theorem can be formulated in terms of measure zero,
then you don't need the apparatus of measure to prove it.

Exercise. If $f:[0,1]\to [0,1]$ is monotone nondecreasing,
then $f$ has a fixed point. 

Special case of a classical theorem.

\begin{theorem}[Knaster-Tarski Theorem]
    Let $L$ be a complete lattice, and let $f:L\to L$ be an
    order-preserving function. Then the set of fixed points of 
    $f$ in $L$ is also a complete lattice.
\end{theorem}

Tarski was a logician, actually he was better than that, he was a
mathematician. Knaster was a topologist.

If $(x_n)\subseteq \RR$ is bounded, then
\[ \limsup_n x_n < \infty. \]
An interpretation of $\limsup$, is that among all convergent
subsequences of $(x_n)$, there exists a subsequence with the 
largest possible limit. (Existence of this subsequence 
is not a priori obvious.)
The one with the largest limit has limit the
$\limsup$ of the sequence.

Want to know what $\sup$ and $\limsup$ are.

Question: Is the exercise also true for nonincreasing functions?
No! (Exercise)

What are interesting questions to ask about fixed points?

How many fixed points can a mapping have? The entire set might be
fixed.

Which sets in $[0,1]$ can be fixed point sets of continuous map?
I.e., the set of
fixed points of some mapping.

\begin{enumerate}[1.]
    \item $[0,1]$ can be such: $f(x)=x$.
    \item Can any finite, nonempty 
        subset in $[0,1]$ be a fixed point set?
        Yes. The proof is a picture. Keep hitting the diagonal.
    \item Can any closed, countable set be a fixed point set?
        Yes, by vote. 
    \item Can any nonempty, closed set be a fixed point set?
        Vote said no, so let's vote again. The answer is yes.
\end{enumerate}

Exercise: The set of fixed points is closed.

Seminar: Cover compactness, completeness, and separability.

Tough exercise: 
For Pierce, let's discuss mappings of $[0,1]^2\to [0,1]^2$.
If the mapping is continuous, it has a fixed point.

Exercises
\begin{enumerate}[1.]
    \item Prove the last list.
    \item Prove the special case.
    \item Prove the Cantor stuff.
    \item Look up the historical people.
    \item Counterexamples to the basic fixed point theorem.
    \item Iterating the logistic function gives the Cantor set.
\end{enumerate}


\end{document}