\documentclass{article}

\usepackage{jragonfyre}
\theoremstyle{remark}
\newtheorem*{question}{Question}
\newtheorem*{fact}{Fact}
\newtheorem{exercise}{Exercise}

\title{Bergelson Notes 6/27}

\begin{document}

\maketitle

\section{Questions}

\begin{exercise}
Prove that any closed, nonempty subset of $I=[0,1]$ is the fixed point set of 
some continuous function $f:I\to I$.
\end{exercise}

\begin{question}
    Can we write a closed set as a union of closed intervals?
    Countable union? Are singletons intervals?
\end{question}

Seminar: Talk about measure. Talk about compactness,
noncompactness, Hilbert space, $\ell^2$. 
Some similarity and some dissimilarity to 
$\RR^n$. Also Cantor sets. Also convex sets, definition and
very basic properties.

\begin{exercise}
    Twist on the previous exercise. Example last time was
    a picture proving any finite set can be the set of fixed 
    points. Also think of the picture of the proof of
    the first fixed point theorem. 

    Another picture is drawn of a single graph with a zero.

    Relationship between zeros of $f(x)-x$ and fixed points
    of $f(x)$.

    Want to consider $g(x):=f(x)-x$.
\end{exercise}

Cantor sets are work of mathematicians, whereas finite sets are 
work of nature. You can't visualize them very easily.

\begin{theorem}
    Given a continuous function $f : I\to \RR$, with
    $f(0)<0$, and $f(1)>0$, there exists $c\in (0,1)$
    such that $f(c)=0$.
\end{theorem}


\begin{theorem}[Intermediate Value Theorem]
    Any continuous function $f : [0,1] \to [0,1]$ has at least 
    one fixed point.
\end{theorem}
\begin{proof}
    Let $g(x)=f(x)-x$. Then $g(0) = f(0)-0 \ge 0$, and
    $g(1)=f(1)-1\le 0$. If we have equality at $0$, or $1$,
    we already have a zero. Otherwise by the intermediate value
    theorem, $g(c)=0$ for some $c\in (0,1)$. Either way, for some
    $c\in[0,1]$, $g(c)=f(c)-c=0$. Then $f(c)=c$, so $c$ is a
    fixed point of $f$.
\end{proof}

One natural generalization of our fixed point theorem is that
if $f:[0,1]\times [0,1] \to [0,1]\times [0,1]$ is a continuous 
function, then $f$ has a fixed point.

One way to prove the generalization might be to complicate
the proof in the $[0,1]$ situation, to give insight into
how to prove the generalization.

\section{The Banach Fixed Point Theorem}

\begin{theorem}[Banach, 1920, in his Ph.D. thesis]
    Let $(X,d)$ be a complete metric space, and assume that
    for some $c\in (0,1)$, a mapping $f:X\to X$ satisfies
    \[ d(f(x),f(y)) \le c \cdot d(x,y) \]
    for every $x,y\in X$,
    then there exists a unique $a\in X$ such that $f(a)=a$.

    $f$ is called a \emph{contraction}. Observe that the definition
    implies that $f$ is continuous.
\end{theorem}

\begin{question}
    What if $c\ge 1$? Is there still such an $a$?
\end{question}

No. Consider the circle $S^1$, and consider the rotation
of the circle, defined by
$\theta \mapsto \theta + \alpha$ for some fixed $\alpha \in\RR$.

If $\alpha \not\in 2\pi \ZZ$, the rotation has no fixed point,
and it is an isometry, that is $d(f(x),f(y)) = d(x,y)$ for all
$x$ and $y$.

\begin{question}
    What are the self-isometries of $[0,1]$?
    We have $f(x)=x$, and $f(x)=1-x$. Are there any others?
\end{question}

\begin{exercise}
    There are only two isometries of $[0,1]$.
\end{exercise}

\begin{question}
    What if you look at $S^1$? We have the rotations
    and reflections. Are there any other isometries?
\end{question}

\begin{exercise}
    Describe all isometries of the circle.
\end{exercise}

\begin{question}
    What if you look at $\RR$ with the Euclidean metric?
    We have shifts. What about flips, like $f(x)=1-x$?
\end{question}

\begin{question}
    What about functions satisfying the definition of contraction
    with $c=1$, with strict inequality?
\end{question}

\begin{definition}
    Let $f:[0,1]\to[0,1]$. $x\in [0,1]$ is $2$-periodic, if 
    $f^2(x) = f(f(x))=x$. More generally $x$ is $n$-periodic,
    if $f^n(x)=x$.
\end{definition}

Suppose we knew there was a $2$-periodic point, is there a point
with period $3$?

Given a function, can we tell if there is an $n$-periodic point.
For self maps of $[0,1]$, there is always a $1$-periodic point,
that is the fixed point theorem. 

This is related to Sharkovsky's Theorem.

\begin{question}
    Is the set of periodic points of a continuous function
    from $[0,1]$ to itself a closed set?
\end{question}

\begin{proof}[Proof of Banach's fixed point theorem]
    First let us prove uniqueness. If $f(a) = a$ and 
    $f(\tilde{a})=\tilde{a}$, then \newcommand\til[1]{\tilde{#1}}
    \[ d(a,\til{a}) = d(f(a),f(\til{a})) \le c \cdot d(a,\til{a}). \]

    Therefore $d(a,\til{a})=0$, so $a=\til{a}$.
    
    Now we prove the existence of $a$. Let $x_0\in X$ be 
    arbitrary, and let $x_1=f(x_0)$, $x_2=f(x_1)$, $\ldots$,
    $x_n=f(x_{n-1})$, $\ldots$. 

    We will estimate $d(x_n,x_{n+1})$. Is it enough to show that
    this distance converges to $0$? 

    \[ d(x_n,x_{n+1}) \le cd(x_{n-1},x_n) \le c^2 d(x_{n-2},x_n)
    \le \cdots \le c^n d(x_0,x_1), \]
    so $d(x_n,x_{n+1}) \to 0$ as $n\to \infty$.

    Recall, $(x_n)\subseteq X$ is a Cauchy sequence if for all
    $\epsilon > 0$, there exists $N \inN$ such that if 
    $n,m > N$, $n< m$, then $d(x_n,x_m) < \epsilon$.

    The sequence is Cauchy because the distance between two 
    adjacent points goes to $0$ exponentially fast.

    \[ d(x_n,x_m) \le \sum_{i=1}^{m-n} d(x_{n+i-1},x_{n+i})
    \le \left(\sum_{i=0}^{m-n-1} c^{n+i} \right)d(x_0,x_1)
    \le c^n d(x_0,x_1)\left(\sum_{i=0}^\infty c^i\right)
    = \frac{c^nd(x_0,x_1)}{1-c}.  \]

    Hence $(x_n)$ is Cauchy, so it converges to some $a\in X$.

    \begin{exercise}
        $f(a)=a$. The proof is $x_{n+1}=f(x_n)$, and $x_{n+1}\to a$,
        $f(x_n)\to f(a)$.
    \end{exercise}
\end{proof}

This gives Picard's theorem from the theory of ordinary
differential equations. We need continuously differentiable
functions to use this theorem. The real correct results assume less,
but we prefer to assume more and get easier to prove results.

It's also used to prove the inverse/implicit function theorem.

\begin{exercise}
    We have some function $f:\RR\to \RR$. We want to find its
    root, so we take a guess $x_0$ close enough to the root.

    Then we define \[ x_1 = x_0 - \frac{f(x_0)}{f'(x_0)},\]
    \[ \vdots \]
    \[ x_{n+1} = x_n -\frac{f(x_n)}{f'(x_n)}. \]

    Try to apply it in special cases. For example, $f(x)=x^2-3$.

    We want to approximate a real number, and find $\frac{p}{q}$
    satisfying
    \[ \abs*{\alpha - \frac{p}{q}} < \frac{1}{cq^2}. \]
\end{exercise}

The problem with this theorem is that the function needs to 
be a contraction.

Suppose we want to solve a system of linear equation 
$A\bar{x} = \bar{b}$. It would be very useful. We would have 
a practical way of finding the zeroes.

\begin{exercise}
There are many metrics on $\RR^n$. We can define
\[ d_p(x,y) = \sqrt[p]{\sum_{i=1}^n |x_i-y_i|^p}.\]
This gives us uncountably many metrics on $\RR^n$, for
$p\in [1,\infty]$. What does it mean when $p=\infty$?

When $p=\infty$, 
\[ d_\infty(x,y) =\lim_{p\to\infty} d_p(x,y)= \max_i |x_i-y_i|.
\]

Prove that these are metrics, and that $d_\infty$ is what is 
claimed.

Three of these metrics are particularly important. 
$d_1(x,y) = \sum_i |x_i-y_i|$ is the taxicab metric.
$d_\infty(x,y)$, the maximum metric, and 
$d_2(x,y)$, the usual Euclidean metric.
\end{exercise}

Most applications of the Banach Fixed Point Theorem are for
functional spaces. These are infinite dimensional spaces, but
there are natural ways of measuring distance on them.

For example $C[a,b]$ is the space of continuous functions on
$f: [a,b]\to\RR$. $\rho(f,g) := \max_{x\in [a,b]} |f(x)-g(x)|$.
This metric makes $C[a,b]$ a complete metric space.

What about $C(\RR)$? Can we give it a reasonable metric?

Can we make matrices over $\RR$ into a metric space?
\[
    M_{n\times n}(\RR)=\set{\text{all $n\times n$ matrices over $\RR$}}
\]
This space is $\RR^{n^2}$ as a vector space, so we can give it 
one of the $d_p$ metrics.

However $M_{n\times n}(\RR)$ has a product.
Can we introduce a \emph{norm} respecting the product of 
matrices.

We can define the norm of vectors in $\RR^n$ by
\[ \|x\|_2 =\sqrt{\sum_i |x_i|^2}.\]

Norms should satisfy
\begin{enumerate}[1.]
    \item $\|x+y\|\le \|x\|\norm{y},$
    \item and $\norm{cx} = \abs{c}\norm{x}.$
\end{enumerate}

What norm can we put on matrices? \emph{Operator norm}.

\begin{exercise}
    Check how to define the \emph{operator norm} on $M_{n\times n}(\RR)$.
\end{exercise}

You can also define norm using eigenvalues. You'll need to replace
$\RR$ with $\CC$, so we have eigenvalues, and consider
$M_{n\times n}(\CC)$.

For every $x\in \RR$, $|x|$ is a norm. $x$ can also be thought of
as the linear map $\RR\to\RR$ defined by $y\mapsto xy$. This
has eigenvalue $x$. Perhaps this suggests how to generalize
norm to $n\times n$ matrices with eigenvalues.

\begin{definition}
    \[ \ell^2(\NN) = \set*{(x_1,x_2,x_3,\ldots) : \sum {x_i}^2 < \infty, x_i\in\RR}\]
    is a complete metric space with respect to the metric
    \[ d(x,y) = \sqrt{\sum_{i=1}^\infty |x_i-y_i|^2} \]
\end{definition}

\section{Weak contractions}

Let's introduce ``weak'' contraction. A weak contraction, is
a function $f:X\to X$ satisfying
\[ d(f(x),f(y))<d(x,y), \]
where $x\ne y$.

\begin{exercise}
    If $X=\RR$, is it true that any weak contraction has a
    fixed point?
\end{exercise}

\begin{question}
    Why don't we ask it about $[0,1]$? Because we already know
    it has a fixed point, independently of it being a weak
    contraction.
\end{question}

\begin{fact}
    If $(X,d)$ is a compact space, and if $f:X\to X$
    satisfies $d(f(x),f(y)) < d(x,y)$ for all $x\ne y$, then
    $f$ has a unique fixed point.
\end{fact}

There are two options, we should be able to use our condition
and compactness to try to force the existence of 
some $c$ smaller than $1$. Or 
we could try to prove it directly using compactness.

\begin{proof}
   \newcommand\til[1]{\tilde{#1}}
   First we prove uniqueness. Let $a$ and $\til{a}$ satisfy
    $f(a)=a$, $f(\til{a}) = \til{a}$. Then if $a\ne \til{a}$, 
    we have
    \[ d(a,\til{a}) = d(f(a),f(\til{a})) < d(a,\til{a}), \]
    which is false. 

    Consider the function $F(x)=d(x,f(x))$. Is it continuous?
    \begin{exercise}
        Prove $F$ is continuous.
    \end{exercise}

    Then since $F$ is continuous on a compact domain, it has
    a maximum and minimum. Let $a$ be a minimum for $F$.

    If $f(a)\ne a$,
    then $F(f(a)) = d(f(a),f(f(a))) < d(a,f(a)) = F(a)$,
    contradiction. Thus $f(a)=a$, so $a$ is a fixed point,
    as desired.
\end{proof}

This proof is great, but its not constructive.

\begin{question}
    If $f:X\to X$ is a weak contraction on a compact space, is
    it a contraction? I.e., is there a constant $c\in (0,1)$
    such that $d(f(x),f(y)) < cd(x,y)$ for all $x,y$?
\end{question}

Given a system of linear equations, and you want a computer
to solve it. If it is small, perhaps we can use Gaussian 
elimination. 

How many operations of multiplication do you have to do to solve
a system of linear equations corresponding to a $n\times n$
matrix. Answer: $\Theta(n^3)$.

\begin{exercise}
    Check that you need $n^3$ operations to solve the system
    by Gaussian elimination.
\end{exercise}

Maybe iteration would be faster.

Can you normalize the matrix, so that iteration yields
a fixed point solving the system of linear equation?

\begin{exercise}
    Assume that $f^{17}$ is a contraction on $X$. Then
    $f$ has a unique fixed point, which is obtainable by 
    iteration. $17$ is an arbitrary positive integer.

    Is such an $f$ necessarily continuous?
\end{exercise}

\begin{exercise}
    $f(x)=\frac{1}{1+x}$, $f(x) = \frac{x}{1+x}$, $x\in [0,1]$.
    Are these weak contractions? Are they contractions?
\end{exercise}

\end{document}