\documentclass{article}

\usepackage{jragonfyre}

%metric spaces 
% 6/24

\title{Metric Space Review with Luke and Evan}
\author{Jason Schuchardt}

\begin{document}
\maketitle

\section{Metric Spaces}

Just as rings are abstractions of the properties of numbers, metric spaces are the abstraction of
the concept of distance.

\begin{definition}
    A \emph{metric space} is a set $X$ of elements, called \emph{points}, equipped with a 
    distance function, $d:X\times X\to \RR$ such that
    \begin{enumerate}[1]
        \item $d(x,x)=0$.
        \item If $x\ne y$, $d(x,y)>0$.
        \item $d(x,y)=d(y,x)$.
        \item $d(x,z) \le d(x,y)+d(y,z)$.
    \end{enumerate}
\end{definition}

\begin{example}
Examples include the line, $\RR$, the plane, $\RR^2$, euclidean space, $\RR^3$, and so on.
Notice that we're ignoring a lot of structure on $\RR^n$.
\end{example}

\begin{example}
Another example is a subspace.

We can measure the distance between points in a subspace using the metric of the ambient space.
% sphere or spiral diagram
\end{example}

\begin{example}
    If $X=\set{\text{subway stations}}$, we can let
     $d(x,y)=\text{length of the shortest path from $x$ to $y$ in the subway network}$
\end{example}

\begin{example}
    The continuous functions on the interval,
    \[ X=C[0,1]:=\set{\text{cts fns $f$}| f:[0,1]\to \RR}.\]

    \[ d(f,g) = \max_{0\le x\le 1} |f(x)-g(x)|\]
\end{example}

Similarly, we have 
\[ X=C^1[0,1]:=
\set{
    \text{continuously differentiable fns $f$}
    |f:[0,1]\to\RR},
    \]
and we can give it the distance
\[ d_{C^1}(f,g) = \max_x\abs{f(x)-g(x)}+\max_x\abs{f'(x)-g'(x)}.
\]

Question: Can we use the $C[0,1]$ metric for continuously differentiable functions?
Answer: Yes, but notice that they can be very different.

Consider $g(x)=\sin(100x)$ and $f(x)=0$ the distance 
$d_C(f,g)\le 1$, whereas $d_{C^1}(f,g) \approx 100$.

\section{Properties of metric spaces}

\begin{definition}
    The (open) \emph{ball} with center $x$ and radius $r>0$ is the set
    \[ B_d(x,r) = \set{y\in X |d(x,y) < r } \]
\end{definition}

\begin{example}
    A ball in $\RR^2$ is a disc,
     and a ball in $\RR^3$ is the interior of a sphere of radius $r$.
\end{example}

\begin{definition}
    A subset $U$ of $X$ is \emph{open} if for all $x\in U$,
    there exists $r > 0$ such that
    \[ B_d(x,r)\subseteq U \]
\end{definition}

\begin{example}
    Open intervals and open balls are open subsets.
\end{example}

\begin{proposition}
    The open ball, $B_d(x,r)$ is open.
\end{proposition}

\begin{proof}
    % diagram of proof
    Let $y\in B_d(x,r)$. By definition $d(x,y)<r$. Let $R=r-d(x,y)$.

    Then we claim that $B_d(y,R)\subseteq B_d(x,r)$. To see this,
    let $z\in B_d(y,R)$. Thus $d(y,z)<R$,
    so 
    \[d(x,z) \le d(x,y) + d(y,z) < d(x,y) + R = d(x,y)+r-d(x,y)=r,\]
    and therefore $z\in B_x(r)$.

    Hence $B_d(y,R)\subseteq B_d(x,r)$ as desired.
\end{proof}

This is why we also call balls open balls.

%\begin{exercise}
%\end{exercise}
Exercise: An arbitrary union of open sets is open, and
a finite intersection of open sets is open.

Note that 
\[ \bigcap_{r>0} B_d(x,r) = \{x\}, \]
and single point sets are not open in $\RR^n$ for example. so
arbitrary intersections of open sets are not generally open.
They may be in weird metrics, like the subway metric.

\begin{definition}
    A subset $F$ of $X$ is \emph{closed} if its complement is
    open.
\end{definition}

Warning! Despite the terminology, it is possible for
sets to be neither open nor closed or even both open and closed.

\begin{definition}
    If $A$ is a subset of $X$, $x\in X$ is a limit point of $A$ if
    for all $\epsilon > 0$, 
    $B_d(x,\epsilon) \cap A$ contains a point other than $x$.
\end{definition}

In other words, every ball around $x$ contains another point 
of $A$. Intuitively, there is no space between $x$ and $A\setminus \{x\}$.

\begin{example}
    The limit points of $(0,1)$ are $[0,1]$, and
    the set
    \[ \set*{\frac{1}{n} : n\in\NN_{>0}} = \set*{1,1/2,1/3,\ldots,1/n} \] 
    has only the limit point $0$.
\end{example}

\begin{definition}
    In contrast to a limit point, an \emph{isolated point} of a subset $A$
    is a point $a\in A$ which is not a limit point of $A$.
\end{definition}


\begin{definition}
    A \emph{Cauchy sequence} in a metric space $(X,d)$ is a sequence
    $\set{x_n}_{n\inN} = x_0,x_1,x_2,\ldots$
     in $X$ such that for all $\epsilon >0$,
    there exists $N \inN$ such that for all $n,m>N$,
    \[ d(x_n,x_m) < \epsilon. \]
\end{definition}

\begin{example}
    $3,3.1,3.14,\ldots$ $\epsilon = 10^{-k}$, $N=k+2$.
\end{example} 

Not a Cauchy sequence: $a_k = \sum_{j\in\NN_{>0}} \frac{1}{j}$.

Exercise: Cauchy sequences are bounded.

\begin{definition}
    A sequence $\set{x_n}_{n\inN}=x_0,x_1,\ldots$ 
    in a metric space $(X,d)$ is \emph{convergent},
    and we say it \emph{converges} to a \emph{limit} $y$
    if for every $\epsilon > 0$, there exists a $N\inN$ such 
    that for all $n > N$, 
    \[ d(x_n,y) < \epsilon. \]
\end{definition}

Exercise: Convergent sequences are Cauchy sequences.

An example with function spaces. 

Question: Is $f_n=x^n \in C[0,1]$ Cauchy?

\begin{definition}
    A metric space $(X,d)$ is \emph{complete} if every Cauchy sequence converges.
\end{definition}

\begin{example}
    $\QQ$ is not complete, but $\RR$ is.
    $(0,1)$ is not complete, but $[0,1]$ is.
\end{example}

Question: Is $C[0,1]$ complete?

\section{Maps of metric spaces}

If $(X,d_X)$ and $(Y,d_Y)$ are metric spaces, we don't want
to consider all maps from $X$ to $Y$, but ones with special
properties.

\begin{definition}
    $f:X\to Y$ is \emph{continuous} if for all convergent
    sequences $x_0,x_1,\ldots \to x$, the image sequence
    \[ f(x_0),f(x_1),\ldots \to f(x). \]
    Intuitively a function is continuous it has no gaps.
    % diagram
\end{definition}

\begin{definition}
    $f$ is a \emph{homeomorphism} if $f^{-1}$ exists 
    both $f$ and $f^{-1}$ are continous.
\end{definition}

Note: If $X$ and $Y$ are \emph{homeomorphic}
(i.e., there is a homeomorphism from one to the other)
then $X$ and $Y$ have the same topology 
(the open sets are the same), but they may have different metrics.
% stereographic projection

\begin{definition}
    We say $f$ is an isometry if $f^{-1}$ exists, and 
    $f$ preserves distances in the sense that for all $x_1,x_2\in X$,
    \[ d_X(x_1,x_2) = d_Y(f(x_1),f(x_2)).\]
\end{definition}

\begin{definition}
    We say $f$ is a contraction if there exists a
    constant $\beta$, with 
    $0<\beta < 1$ such that
    $f$ shrinks distances in the sense that for all $x_1,x_2\in X$,
    \[  d_Y(f(x_1),f(x_2)) < \beta d_X(x_1,x_2).\]
\end{definition}

Note that contractions and isometries are continuous.

\begin{theorem}
    A contraction $f:X\to X$ on a complete metric space $(X,d)$ has
    a unique fixed point. I.e.
    there is a unique point $x\in X$ such that $f(x)=x$.
\end{theorem}

\end{document}