\documentclass{article}

\usepackage{jragonfyre}
\title{7/6 Metric Space Completion and Universal Properties with Zachary}
\newcommand\oln\overline 
\newcommand\uln\underline 

\begin{document}

We'll talk a bit about $\RR$ and completions of metric spaces.

\section{Why $\RR$?}

$\QQ$ has ``holes.'' 

Problem: There are Cauchy sequences in $\QQ$ with 
no limit.

For example, we might consider the sequence 
\[(3,3.1,3.14,3.141,3.1415,\ldots).\]
This is a Cauchy sequence with no limit in $\QQ$.
So where does the limit live? Answer: It lives in $\RR$.

What is $\RR$, what properties should it have?
\begin{itemize}
    \item $\RR$ should be \emph{complete}, i.e., all 
        Cauchy sequences in $\RR$ should converge to a
        limit in $\RR$.
    \item $\QQ$ should be \emph{dense} in $\RR$, 
        meaning that its closure $\oln{\QQ}$ should be 
        all of $\RR$.
    \item $\RR$ should be a \emph{field}.
    \item $\RR$ should be \emph{totally ordered}, meaning 
        that for every $x,y\in\RR$, either 
        $x\le y$ or $y\le x$.
\end{itemize}

Now $\RR$ should be a metric space. However, when we define
a metric, we usually take a metric to be a map $X\times X\to \RR$.
However, we're trying to construct $\RR$, so we have to 
adjust things a bit.

There are many ways to construct $\RR$ from $\QQ$. Some of them 
are historical, and we don't use them much anymore, like the 
Dedekind cut construction.

Instead, we'll construct the metric space completion of $\QQ$.

\begin{definition}
    Let $(X,d)$ be a metric space, $d:X\times X\to \RR$. 
    We look at 
    Cauchy sequences in $X$.
    Let $a=(a_n)_{n\ge 1}$, 
    $b=(b_n)_{n\ge 1}$ be Cauchy sequences.
    Define $a\sim b$ if $d(a_n,b_n)_{n\ge 1}$ is Cauchy with 
    limit $0$.

    This is an equivalence relation, and the equivalence 
    classes of the equivalence relation will form the set 
    underlying $\tilde{X}$, the completion.
\end{definition}

However, we don't have a notion of $\RR$ yet, so let's just 
think about $\QQ$ and forget about general metric spaces. 

We have a map $d:\QQ\times \QQ\to \QQ$ defined by 
$(x,y)\mapsto \abs{x-y}$.

Once again, we can consider Cauchy sequences in $\QQ$.
We say $a\sim b$ if \[ (|a_n-b_n|)_{n\ge 1} \to 0. \]
Then $\RR$ as a set is defined to be $C(\QQ)/\sim$, where 
$C(\QQ)$ denotes the set of Cauchy sequences in $\QQ$.

Now we get a metric structure on $\RR$, $d:\RR\times \RR\to \RR$
defined by 
\[ d([a],[b]) = [(|a_n-b_n|)_{n\ge 1}] \]

%\begin{exercise} TODO
    $\RR$ is complete with respect to $d$.
%\end{exercise}

Do we have a field structure on $\RR$?
I.e., do there exist maps $+:\RR\times \RR\to\RR$, 
$\cdot : \RR\times \RR\to \RR$, $- : \RR\to\RR$,
and $\cdot ^{-1} : \RR\setminus\set{0} \to \RR\setminus\set{0}$
that define a field structure on $\RR$.
Can we get these maps from the corresponding maps on $\QQ$?

Answer: yes.

\section{Digression to Category Theory}

Or perhaps, the rest of math is a digression from category theory.

Question: What is a category?
Categories consist of objects and things that are 
called either arrows, morphisms, or maps.

For example, we have the category $\Ring$, whose objects 
are rings and whose arrows are ring homomorphisms.
We draw diagrams of these objects and arrows like the following.
\[ 
\begin{tikzcd}
R \arrow[rd, "g\circ f"'] \arrow[r, "f"] & S \arrow[d, "g"] \\
                                         & T               
\end{tikzcd}
\]
The objects are rings $R$, $S$, and $T$, and the morphisms 
are $f$, $g$, and $g\circ f$. We say that the diagram commutes,
because the compositions along all the possible paths in the 
diagram is the same. Either we travel along $f$ and then $g$,
and get $g\circ f$, or we directly travel along $g\circ f$,
and also get $g\circ f$.

\begin{definition}
    A \emph{category}, $\calC$ is a collection of objects
    and morphisms between them. Between any two objects 
    $X$ and $Y$, there is a set $\Hom_{\calC}(X,Y)$.
    For any three objects, $X,Y,Z$, there is a composition law, 
    \[ \circ : \Hom_{\calC}(Y,Z)\times \Hom_{\calC}(X,Y) \to 
    \Hom_{\calC}(X,Z).\]
    The composition law is required to be associative.

    Furthermore, for every object $X$, there is an element
    \[\id_X \in \Hom_{\calC}(X,X).\]
    The identity morphisms are identities for the composition
    law.
\end{definition}

While this definition may seeem a bit complicated, we 
can create categories very easily. For example, we can
draw the diagram 
\[X\to Y,\]
which is a category with three morphisms (the two identity maps,
$\id_X$, $\id_Y$
are omitted in this diagram).

For another example, let $X$ be a set. Then 
the set $\powerset{X}$ can be viewed as a category.
The objects are subsets of $X$, and the morphisms 
are the inclusiom maps (when they exist).

I.e., if $A,B\subset X$, then if $A\subset B$, we have a map
$i:A\to B$ defined by $i(a)=a$. If $A\not\subset B$, then 
there is no morphism between $A$ and $B$.

This last example, will be particularly helpful for us.
We will want a \emph{full subcategory}.

\begin{definition}
    Let $\calC$ be a category. Then a \emph{full subcategory}
    is a subcategory, such that between two objects in the 
    subcategory, I have all possible morphisms.
\end{definition}

Let $X$ be a metric space. Let $\calC$ be the full
subcategory of $\powerset{X}$ whose objects are closed subsets.

\begin{definition}
Given $A\subset X$, a closure of $A$ (if it exists) 
is a closed subset $B$ of $X$
such that if $D\subset X$ is closed and contains 
$A$, then $D$ contains $B$.
\end{definition}

For example, the closure of $(0,1)$ in $\RR$, should be $[0,1]$.

Claim: The closure $B$ of $A$ exists and is unique.

\begin{proof}
    We start with uniqueness. Suppose that $B$ and $B'$ are 
    both closures of $A$. Then $B\subset B'$ and $B'\subset B$,
    so $B=B'$.

    Now for existence, follows from the fact that arbitrary
    intersections of closed points is closed, so 
    \[ B = \bigcap_{F\supseteq A} F \]
    is closed, contains $A$, and is a subset of every 
    closed set containing $A$.
\end{proof}

For every $A\subseteq X$, we have a unique closed set 
$\bar{A}\subset X$. Thus we have a mapping of objects 
from $\powerset{X}$ to $\calC$. Since $A\subset B$ implies 
$\bar{A}\subset \bar{B}$, this mapping also induces a mapping 
of morphisms between the corresponding objects.
Thus this is an example of a \emph{functor}.

\begin{definition}
    If $\calC,\calD$ are categories, a function $F:\calC\to\calD$
    is a mapping of objects in $\calC$ to objects in $\calD$,
    together with maps 
    \[\Hom_{\calC}(X,Y)\to \Hom_{\calD}(FX,FY),\]
    for every pair of objects $X$ and $Y$ in $\calC$.
    Each of these maps of morphisms is also written as $F$.

    We require these maps to preserve composition in the following 
    way.
    I.e., if $X,Y,Z$ are objects of $\calC$, $f:X\to Y$, $g:Y\to Z$,
    then we require \[ F(g\circ_\calC f) = Fg\circ_\calD Ff. \]
\end{definition}

Now you can verify that closure is a functor as described before.
Moreover, closure is a basic example of a solution to a 
universal problem.

Given any $A\subset X$, $B$ closed, there is a unique inclusion
map filling in the dashed line in the diagram.
\[
\begin{tikzcd}
                                                                   & B                                                             \\
A \arrow[ru, "\text{incl.}", hook] \arrow[r, "\text{incl.}", hook] & \overline{A} \arrow[u, "\exists!\text{incl.}"', dashed, hook]
\end{tikzcd}
\]

\section{Returning to completion}

Let \newcommand\Met{\mathbf{Met}} $\Met$ be the category of 
metric spaces with uniformly continuous maps as morphisms.
Let \newcommand\Comp{\mathbf{Comp}} $\Comp$ be the full 
subcategory of complete metric spaces.

We want to solve the universal problem of completion.
I.e., given $X\in \Met$, we want $\hat{X}\in\Comp$, 
with a map $i:X\to\hat{X}$,
such that
for every $Y\in\Comp$, and $f:X\to Y$, there is a unique 
map $h : \hat{X}\to Y$ such that the following diagram commutes:
\[
\begin{tikzcd}
                                 & Y                                       \\
X \arrow[ru, "f"] \arrow[r, "i"] & \hat{X} \arrow[u, "\exists!h"', dashed]
\end{tikzcd}
    \]

Why do we impose uniform continuity? % TODO

Claim: $\hat{X}$ exists.
This is constructed as indicated before. I.e., we take 
Cauchy sequences in $X$, and impose the same equivalence
relation as before. Then the points of $\hat{X}$ are the 
equivalence classes of these Cauchy sequences in $X$.

If the metric on $X$ is $d$, then we have 
the distance on $\hat{X}$
\[ \hat{d}(a,b):= \lim_{n\to\infty} d(a_n,b_n). \]
You can then check that everything exists and is well defined.

Now why do we care about category theory? Well, we'd like 
everything we do to be independent of the construction.

Let $\Met'$ be the category of metric spaces with morphisms 
given by the isometries. Let $\Comp'$ be the same but 
for complete metric spaces. Then if we solve the same 
problem again, we will get the same solution. 

Apply this to $\QQ$. Let $\phi:\QQ\to \RR$ be the solution 
to the universal problem. This is injective, since it is 
an isometry by the fact that this solves the second universal
mapping problem. Thus $\RR$ contains an isomorphic copy of $\QQ$.
Moreover, $\QQ$ is dense by the construction. What about the 
field structure?

We have the map \[ +:\QQ\times \QQ \to \QQ \into\RR. \]
This map is uniformly continuous. Then 
note that $\widehat{\QQ\times \QQ} \simeq \RR\times \RR$.

Thus $+$ extends to $\RR\times \RR$ by the universal property 
of the completion.

\end{document}