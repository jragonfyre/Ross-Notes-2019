\documentclass{article}

\title{Monstrous Menagerie with Vandehey 6/26}
\author{Jason Schuchardt}

\usepackage{jragonfyre}
\newcommand\cft[1]{\frac{1}{#1+}}
\newcommand\hoint{[0,1)}
\newcommand\recip[1]{\frac{1}{#1}}

\begin{document}

Who here knows the most digits of $\pi$?

\[ \pi \approx 3.14159\cdots \]

At the $100$th place in $\pi$, we see $10$ $0$s in a row?

At the $7,950,431$st place in $\pi$, we see $10$ $0$s in a row?

We sort of think the second one is slightly more reasonable? Why do we think that?

What do we know about the digits of $\pi$?

How many $7$s are in the decimal expansion of $\pi$?

Infinitely many. Positive density.
$7$ should come up $1/10$th of the time.

All the statements that just came up are unknown!

Why is this a hard question? $\pi$ is a natural object, what is the unnatural thing to do to it?
Writing it in decimal notation.

Why base 10? It's social convention.

In base 10, multiplying by $10$ is really easy. It's $O(1)$. Multiplying by $3$? It's fairly easy,
but nontrivial.

How about taking square roots, or reciprocals? Not so easy to do.
If we move to base 3, multiplying by 3 is trivial, but multiplying by 10 is fairly easy
but nontrivial.

\section{Continued Fractions}

The golden mean, 
\[ \phi = \frac{1+\sqrt{5}}{2} =  1+\frac{1}{1+\frac{1}{1+\frac{1}{1+\frac{1}{1+\cdots}}}} \]
This is a continued fraction expansion.

How many digits of $e$ do you know? I know them all. 

\[e=2+\frac{1}{1+}
\frac{1}{2+}\frac{1}{1+}\frac{1}{1+}\frac{1}{4+}\frac{1}{1+}\frac{1}{1+}\frac{1}{1+}\frac{1}{6+}
\cft{1}\cft{1}\cft{8}\cft{1}\cft{1}\cft{10}\cft{1}\cdots
\]

Where does this come from? Who knows, it came from Ramanujan's notebooks.

How do we multiply by $10$ in this representation? It's not easy.

Let's consider some rules that we take for granted that we can break.

\section{Broken Rules}

If I add together two numbers with finite expansions, I expect the result to have a finite number of
digits. Not true in general.

We know all the numbers that have periodic expansions. Rational numbers. That doesn't hold in general.

It is obvious that an infinite base 10 expansion converges. Maybe?

There are actually two questions here. How do we express integers vs how do we express real numbers?
We'll stick to real numbers, since those are a bit harder.


\section{The Monstrous Menagerie}

\subsection{Base-$b$ expansions}

Any number $x\in \hoint$
can be written as 
\[ \frac{a_1}{b} + \frac{a_2}{b^2}+\cdots = \sum_{i=1}^\infty \frac{a_i}{b^i} \]
where $a_i\in \set{0,1,2,\ldots,b-1}$, with $\infty$ many $a_i$ not equal to
$b-1$.

\subsection{$Q$-cantor series expansion}

$Q=(q_1,q_2,q_3,\ldots)$ with $q_i\ge 2$ a sequence of integers.
Any $x\in \hoint$ can be written as 
\[\frac{a_1}{q_1} + \frac{a_2}{q_1q_2} + \frac{a_3}{q_1q_2q_3} + \cdots = 
\sum_{i=1}^\infty \frac{a_i}{\prod_{j=1}^i q_j}\]
with 
$a_i\in [0,q_i-1]$ and $\infty$ many times $a_i\ne q_i-1$.

We have 360 degrees in a circle because the Babylonians used mixed base,
alternating $6$, $10$, $6$.

\subsection{$\beta$-expansion}

$\beta>1$ is a real number.

Any $x\in\hoint$ can be written as
\[ \frac{a_1}{\beta} + \frac{a_2}{\beta^2} + \frac{a_3}{\beta^3}+\cdots
=\sum_{i=1}^\infty \frac{a_i}{\beta^i}\]
where $a_i\in [0,\ceil{\beta}-1]$.

$\beta$ expansions tend to go really, really wrong. Bad things 
happen.

For example, consider $\beta = \phi = \frac{1+\sqrt{5}}{2}$.

$\beta^2 = \beta + 1$, so
\[ 1 = \frac{1}{\beta} + \frac{1}{\beta^2},\]
or in other words.
So
\[ 1_\beta = 0.11_{\beta} = 0.1011_\beta = 0.101011_\beta =
\cdots 
= 0.\overline{10}_\beta\]

For a typical $\beta$, almost every number has an uncountable 
number of $\beta$ expansions.

\subsection{Balanced Ternary}
Each $x\in \left[\frac{-1}{2},\frac{1}{2}\right)$ can be written
as 
\[ \frac{a_1}{3} + \frac{a_2}{3^2} + \frac{a_3}{3^3} + \cdots
= \sum_{i=1}^\infty \frac{a_i}{3^i} \]
with $a_i\in\set{-1,0,1}$.

This has a bit of interesting history. When the Soviets were 
building computers, they initially thought about using
balanced ternary rather than binary, since it would result
in fewer carries. However, this proved to be unfeasible.

There is also base $-10$, with digits $[0,9]$, known as 
\emph{negadecimal}.

\subsection{Base $i-1$}

\[ \sum_{n=1}^\infty \frac{a_n}{(i-1)^n}, \]
$a_n\in [0,1]$.

The numbers you can represent this way belong to the twin dragon
fractal, and it is sometimes also known as the twin dragon 
expansion.

\subsection{Engel series}

$x>0$ can be written as 
\[ \frac{1}{a_1} + \frac{1}{a_1a_2} + \frac{1}{a_1a_2a_3} + \cdots 
= \sum_{i=1}^\infty \frac{1}{\prod_{j=1}^i a_j }, \]
where the $a_i$ are a sequence of non-decreasing integers.

\subsection{Sylvester series}

$x\in \hoint$
\[ \frac{1}{a_1}+\frac{1}{a_2} + \frac{1}{a_3} +\cdots \]
where $a_{i+1} \ge a_i(a_i-1) + 1$.

\subsection{L\"uroth series}

$x\in\hoint$
\[ \frac{1}{a_1} + \frac{1}{a_1(a_1-1)a_2} + 
\frac{1}{a_1(a_1-1)a_2(a_2-1)a_3} + \cdots
=\sum_{i=1}^\infty \frac{1}{\prod_{j=1}^i a_j \prod_{j=1}^{i-1}(a_j-1)} \]
$a_i\inN$, $a_i \ge 2$.

Every rational has a periodic L\"uroth series.

\subsection{Regular CF}

$x\in \hoint$

\[ \cft{a_1}\cft{a_2}\cft{a_3} \cdots, \]
where $a_i \inN$.

Expression is finite if $x\in\QQ$.

\subsection{Even CF}

\newcommand\cftt[2]{\frac{#1}{#2+}}

\[ \cft{a_1}\cftt{e_1}{a_2}\cftt{e_2}{a_3}\cftt{e_3}{a_4}\cdots\]
$a_i\in 2\NN$, $e_i\in \set{\pm 1}$.

There is also the odd CF, and many other kinds of CF, including
grotesque CF.

Recent paper was working on CF for quaternions, and octonions.

They've also been done for polynomials and $p$-adic numbers.

What would be an application of even CFs? 

There are Gauss sums
\[ \sum_{n=1}^N e^{2\pi i \alpha n^2}, \]
and if $\alpha$ is an irrational number, this will not repeat,
and it will create crazy spirals. 

If you zoom out and connect the centers of the spiral together,
it creates another spiral of the exact same type. You get a 
fractal nesting of spirals.

The even CF expansion of $\alpha$ encodes information about
this nesting of spirals.


What's the least interesting CF? 
The mother of all continued fraction expansions. Has highly
restricted $a_i$ and $e_i$. It contains all other
CFs as subexpansions.

\section{How to find digits?}

base $10$ expansion of $\sqrt{2}$. Know $1< \sqrt{2} < 10$.
So first nonzero digit is in one's place.

Is $1\overset{?}{<} \sqrt{2}$, yes, because $1^2=1<2$.
$2\overset{?}{<} \sqrt{2}$, no, so the one's digit is $1$.




\end{document}