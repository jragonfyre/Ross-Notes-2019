\documentclass{article}


\title{Type Theory with Paige North 7/9}
\author{Jason Schuchardt}

\usepackage{jragonfyre}
\usepackage{turnstile}
\usepackage{bussproofs}
\usepackage{booktabs}
\newcommand\blank{ \underline{\phantom{$\quad\quad$}} }
\newcommand\TYPE{\text{ \scshape{type}}}
\newcommand\ctxt{\text{ ctxt}}
\newcommand\add{\operatorname{add}}
\newcommand\mult{\operatorname{mult}}
\newcommand\LIST{\textsc{List}}
\newcommand\nil{\textrm{nil}}
\newcommand\con{\operatorname{con}}

\begin{document}

\maketitle

\section{Set interpretation of type theory}

We were talking about the set interpreation of type theory 
last time. See a summary in Table \ref{tab:set-corr}.

\begin{table}[h]
    \centering 
    \begin{tabular}{ll}
        \toprule
        Type & Set \\
        \midrule
        Term & Element \\
        Dependent term, $x:T \yields s(x):S$
        & Function $T\to S$\\
        $S\implies T$ & Set of functions,
        $\Hom(S,T)$\\
        $S\wedge T$ & Cartesian product, $S\times T$ \\
        $S\vee T$ & Disjoint union, $S\sqcup T$ \\
        $\bot$ & $\nullset$\\
        $\top$ & $\set{*}$\\
        $\lnot A:=A\implies \bot$ & $\Hom(A,\nullset)$ \\
        $A\vee \lnot A$ & $A\sqcup (A\to \nullset) = 
        \begin{dcases*}
            \set{*} & if $A=\nullset$ \\
            A & otherwise
        \end{dcases*}$ \\
        \bottomrule
    \end{tabular}
    \caption{The correspondence between type theoretical notions
    and their interpretations in set theory.}
    \label{tab:set-corr}
\end{table}

\section{Natural Numbers}

What should they be?

In any type definition, we introduced terms.
For example, we had 
%\begin{prooftree}
    %\AxiomC{}
    %\UnaryInfC{$\yields * : \top$}
%\end{prooftree}
\[\frac{}{\yields * : T}\quad 
\frac{\yields a : A}{\yields i_1(a):A\vee B}
\quad \frac{x:S\yields t:T}{\yields \lambda x. t : S\implies T}
\]
These elements are called canonical terms. What are the canonical
terms of $\NN$?

\begin{definition}
    The natural numbers type will be defined by the following 
    rules:
    \begin{enumerate}
        \item The natural numbers is a type:
        \[\frac{}{\NN\TYPE}\]
        \item Term construction
        \[
        \quad \frac{}{0:\NN}
        \quad \frac{\Gamma\yields n:\NN}{\Gamma\yields sn :\NN}.
        \]
        \item We also need a way to access the terms in the type,
            which we will do by specifying how we can build 
            functions out of the type. This is like with the 
            $\vee$ type, where we had the rule 
            \begin{prooftree}
                \AxiomC{$\Gamma,a:A\yields y:Z$}
                \AxiomC{$\Gamma,b:B\yields z:Z$}
                \BinaryInfC{$\Gamma, c:A\vee B
                \yields j_{y,z}(c) : Z$}
            \end{prooftree}
            For the natural numbers, we have
            the rule (which we might call induction)
            \begin{prooftree}
                \AxiomC{$T\TYPE$}
                \AxiomC{$\Gamma\yields z:T$}
                \AxiomC{$\Gamma,t:T\yields \sigma t:T$}
                \TrinaryInfC{$\Gamma,n:\NN\yields j_{z,\sigma}(n):T$}
            \end{prooftree}
        \item We need this to satisfy the following equality rules:
            \[\Gamma\yields j_{z,\sigma}(0) = z :T,\]
            and 
            \[\Gamma, n:\NN\yields j_{z,\sigma}(sn) = 
            \sigma(j_{z,\sigma}(n)) : T.\]
    \end{enumerate}

    We say $\NN$ is \emph{inductively} or \emph{recursively}
    defined.
    $\NN$ is defined to be the type whose terms are $0:\NN$,
    $sn:\NN$ for every $n$.
\end{definition}

Contexts were also recursively defined, as a list, with the 
rules:
\[\frac{}{\nullset\ctxt}\quad
\frac{\Gamma \ctxt\quad T\TYPE}{\Gamma,x:T}\]

The natural numbers form a sort of tree, as do contexts.
\[
    \begin{tikzcd}
0 \arrow[d, no head]    & \nullset \arrow[d, no head]                                 & \Gamma                \\
s0 \arrow[d, no head]   & x:\Gamma \arrow[d, no head] \arrow[ru, no head]             & \Delta                \\
s^20 \arrow[d, no head] & {x:\Gamma, y:\Delta} \arrow[d, no head] \arrow[ru, no head] & E \arrow[ld, no head] \\
s^30                    & {x:\Gamma, y:\Delta, z:E}                                   &                      
\end{tikzcd} 
    \]
On the left, we have the construction of $3:\NN$, and on the right,
we have the construction of the context 
$x:\Gamma,y:\Delta,z:E$.

The types in the simply typed lambda calculs 
are also recursively defined.
We started with $T_1,\ldots,T_n$.
For example,
the tree for $(T_1\implies T_2)\implies (T_3\implies T_4)$ is
\[
\begin{tikzcd}
T_1 \arrow[d, no head]              & T_2 \arrow[ld, no head]                    & T_3 \arrow[d, no head]              & T_4 \arrow[ld, no head] \\
T_1\implies T_2 \arrow[rd, no head] &                                            & T_3\implies T_4 \arrow[ld, no head] &                         \\
                                    & (T_1\implies T_2)\implies(T_3\implies T_4) &                                     &                        
\end{tikzcd}
    \]

\subsection{Examples of using the function construction rule for 
the natural numbers}

We'll construct a function 
\[ f:T\implies T, n:\NN \yields f^n :T\implies T,\]
where $f^n=f\circ f \circ f \circ \cdots \circ f$ $n$ times.

The plan is to define $c(f,n)$ by the rules 
$c(f,0)=\id_T$ and $c(f,sn) = f\circ c(f,n)$.

The value at zero is 
\[ f:T\implies T \yields 
    \lambda x.x : T\implies T, \]
and the inductive step is 
\[f:T\implies T, g:T\implies T \yields 
        f\circ g : T\implies T.
    \]
These yield
\[ f:T\implies T,n:\NN \yields 
    j_{\lambda x.x,f\circ -}(n) : T\implies T.
    \]

Then we know that $j(f,0)=\lambda x.x$, and 
$j(f,sn) = f\circ j(f,n)$.

\begin{question}
    It seems like we could do the composition in the other 
    order, and its not clear that they are equal.

    Answer: Yes. We need a stronger type theory to prove that.
    We can prove it for a specific number, e.g.,
    we can prove $f\circ (f\circ f) = (f\circ f) \circ f$.
\end{question}

\begin{example}
    Define the function $\lambda x.s^2 x : \NN\to \NN$
    using induction rather than lambda abstraction.
    (This is the function $x\mapsto x+2$)

    For $0$, we have
    \[\yields s(s(0)), \]
    and for the inductive step, we have
    \[ n:\NN \yields sn :\NN, \]
    so applying the inductive function formation rule, we have 
    \[n:\NN \yields j_{s^20,s}(n) : \NN.\]

    Checking, we have $\yields j(0) = s^20:\NN$,
    and $n:\NN \yields j(sn) = s(jn) : \NN$.

    Question: It seems like we could prove that this function
    is actually equal to $\lambda x.s^2x$ with some sort of 
    induction. Would that be a function from $\NN$ to some 
    equality type?

    Answer: Yes, but we don't have an equality type yet.
    Hopefully we'll talk about the dependently typed lambda 
    calculus tomorrow, and then we can introduce that type.
\end{example}

\begin{example}
Want to define $n:\NN, m: \NN\yields \add(n,m):\NN$.
For zero we have 
\[n:\NN\yields n:\NN, \]
and for the inductive step, we have
\[n:\NN,x:\NN \yields sx : \NN. \]
This yields 
\[n:\NN,m:\NN \yields j_{n,\lambda x.sx}(m) : \NN. \]

Checking this, we have $n:\NN \yields j_n(0) = n$,
and $n:\NN,m:\NN \yields j_n(sm) = sj_n(m) : \NN$.

Notice the asymmetry here. We inducted on $m$, and we could have
inducted on $n$. Metatheoretically, we can see that these two 
ways of defining addition are the same, but hopefully 
next time, we can prove that they are the same inside the 
type theory.
\end{example}

\begin{example}
    Now we can define multiplication!
    $n:\NN, m:\NN \yields\mult(n,m):\NN$.

    Once again, we start with $0$:
    \[ n:\NN \yields 0:\NN, \]
    and the inductive step:
    \[ n:\NN,x:\NN \yields \add(x,n).\]
    Then by induction, we get the function
    \[ n:\NN, m:\NN \yields \mult(n,m):\NN, \]
    and we know that 
    $\mult(n,0)=0$, and $\mult(n,sm) = \add(\mult(n,m),n)$.
\end{example}

\section{The list type}

Lists are defined by the rules:
\begin{enumerate}
    \item \[\frac{T\TYPE}{\LIST(T)\TYPE}\]
    \item Canonical elements:
        \[\frac{}{\nil:\LIST(T)},\quad 
        \frac{\Gamma \yields \ell:\LIST(T),\ t:T}{
            \Gamma \yields \con(\ell,t) : \LIST(T)
            }
        \]
    \item Induction:
        \[ \frac{\Gamma\yields s:S,\quad 
            \Gamma, x:\LIST(T), y:T\yields c(x,y):S }{
                \Gamma,\ell:\LIST(T)\yields j_{s,c}(\ell):S
            }
            \]
    \item Where induction satisfies the coherence rules:
        \[j_{s,c}(\nil)=s,\text{ and } j_{s,c}(\con(x,y))=c(x,y)
        \]
\end{enumerate}

\end{document}