\documentclass{article}


\title{Type Theory with Paige North 7/19}
\author{Jason Schuchardt}

\usepackage{jragonfyre}
\usepackage{turnstile}
\usepackage{bussproofs}
\usepackage{booktabs}
\newcommand\blank{ \underline{\phantom{$\quad\quad$}} }
\newcommand\TYPE{\text{ \scshape{type}}}
\newcommand\ctxt{\text{ ctxt}}
\newcommand\add{\operatorname{add}}
\newcommand\mult{\operatorname{mult}}
\newcommand\List{\textsc{List}}
\newcommand\nil{\textrm{nil}}
\newcommand\con{\operatorname{con}}
\newcommand\pred[1]{\operatorname{#1}}
\newcommand\isEquiv{\pred{isEquiv}}
%\newcommand\IdT{\mathrm{Id}}
\newcommand\refl{\mathrm{refl}}
\newcommand\BB{\mathbb{B}}
\newcommand\Rel{\operatorname{Rel}}
\newcommand\FunExt{\mathrm{FunExt}}
\newcommand\isContr{\pred{isContr}}

\begin{document}

\maketitle

\section{Identity types}

Last time we talked about identity terms in $\Sigma$-types,
and we gave an explicit description. Can we do this for 
other types? Unfortunately, in general this isn't possible 
for the other types. There are models of type theory,
in which the things we'd want to be the same aren't the 
same. So we postulate this for our type theory, since we 
don't care about those models.

\subsection{$\Pi$ types}

\[\Id_{\prod_{x:B}E(x)} (f,f') \simeq ? \]
We already talked about homotopy,
\[ f\sim f' := \prod_{x:B} \Id(fx,f'x). \]
We have 
\[\pred{idToHo}:\prod_{f,f':\prod_{x:B}E(x)} \Id(f,f') \to (f\sim f').\]
How does this work?
By induction, we need to produce 
\[\prod_{f:\prod_{x:B}E(x)} (f\sim f) 
= \prod_{f}\prod_{x:B} \Id(fx,fx),\]
and we already have a term in this last 
type,
\[\lambda f,x. \refl_{fx} .\]

We would hope that there is a function in the other direction.
I.e., we want a term of type 
\[\isEquiv(\pred{idToHo}).\]
This doesn't exist in general, so we postulate the existence 
of such a term,
\[ \FunExt_{f,f'} : \isEquiv(\pred{idToHo}_{f,f'}). \]
This is called function extensionality.

If you are interested in bad models, which don't satisfy this,
you can check out: von Glehn. Dialectica Models of TT.

\subsection{$\Id$ types}

We could postulate the following:
\[\pred{UIP}:\Id_{\Id(s,t)}(p,q) \simeq \top \]
UIP stands for uniqueness of identity proofs.
However, we definitely don't assume this one.
It's not equivalent to assuming axiom K that 
we talked about before, but it's morally the
same, so we don't want to assume this either.

\subsection{$U$ types}

\[\Id_{U}(S,T) \simeq ?\]
Should $?$ be $S\simeq T$?

Just like with $\Pi$-types, we can produce a map 
\[ \pred{idToEquiv} : \prod_{S,T:U} \Id_U(S,t) \to (S\simeq T).\]
We can postulate 
\[\pred{UA} : \isEquiv(\pred{idToEquiv}),\]
and this axiom is called the univalence axiom.

Univalence implies function extensionality. 
However, univalence and uniqueness of identity proofs are 
incompatible. If you assume both, you can prove false.

\section{Programs}

Mathematicians and theoretical CS people are interested in 
the following two programs.

\subsection{Univalent foundations}

\begin{enumerate}
\item It's a new foundation of mathematics.
\item It can be formalized/computer-checked.
\item Dependent type theory with $\Sigma$, $\Pi$, $\Id$, $\top$,
$\bot$, $\BB$, $\NN$, $\pred{UA}$, propositional truncation, 
and propositional resizing. (These last two we haven't seen, 
and we won't talk about the last, since it's not clear that 
it's consistent). 
\item Invented by Vladimir Voevodsky. He won a Fields 
medal for work in motivic homotopy theory. His proofs were found 
to have a lot of holes, though he managed to fix the ones found 
in his Fields medal work.
\item UniMath (GitHub)
\end{enumerate}

\subsection{Homotopy Type Theory}

\begin{enumerate}
    \item Basically the same, but not purporting to be a new 
        foundation for mathematics.
    \item DTT with $\Sigma$, $\Pi$, $\Id$, certain (higher)
        inductive types, and univalence.
    \item Emphasizes the connections between the type theory
        and classical homotopy theory.
\end{enumerate}

\section{$h$-levels}

$h$ stands for homotopy. Consider a type $T$.
It has the terms $r,s,t:T$. Think of $T$ as a space, with 
$r,s,t$ points in the space.
It has the paths $p,q:\Id(s,t)$. I.e., terms of the identity type 
represent paths between the points in the space.
It also has paths $\alpha,\beta : \Id(p,q)$. We can think of these
as deformations of paths in the space. I.e., these are 
homotopies of paths.

Write $\Id(s,t)$ as $s=t$, and $s=t$ as $s\equiv t$ now.

Stratify types into their homotopy levels.
\begin{enumerate}
    \item[Level 0:] Types $\simeq \top$.
    \item[Level 1:] Types $\simeq \bot$ or $\top$.
    \item[Level 2:] Types that look like sets. (Types with terms)
    \item[Level 3:] Types that look like graphs. (groupoids) (Types with terms and paths, but no paths between paths)
    \item[Level 4:] $2$-groupoids. 
\end{enumerate}

Now let's be rigorous.

\subsection{$h$-level $0$ (contractible)}

Some people say that this is level $-2$ to make the numbering work out 
with the way we number groupoids. 
\[T:U\yields \isContr(T):U,\]
where 
\[\isContr(T) := \sum_{t:T} \prod_{s:T} t=s. \]

\begin{proposition}
    $\top$ is contractible.
\end{proposition}

\begin{proof}
    Take $*:\top$, then we want $?:\prod_{x:\top} *=x$.
    By induction, we can assume $x\equiv *$ by the elimination 
    rule for $\top$. Then we have $\refl_* : *=*$.
    This gives us $\ds j_{\refl_*} : \prod_{x:\top} *=x$, as 
    desired.
\end{proof}

Thus up to homotopy, $\top$ is a one point space.

\begin{proposition}
    We have a map $\isContr(S)\to (S\simeq \top)$ (actually this 
    is an equivalence).
\end{proposition}

\begin{proof}
    Consider 
    $(t,p):\isContr(S)$, where $t:S$, and $p:\prod_{x:S} t=x$.
    There is a function $g:S\to \top$, given by 
    $\lambda s. *$. There is a function $f:\top\to S$ by 
    $*\mapsto t$.

    We have $g\circ f \sim \id_\top := \prod_{x:\top} gfx = x$,
    which by induction,
    we can prove by proving on the canonical term, $*$.
    On the canonical term, we have $gf* \equiv gt \equiv *$,
    and so we have $\refl_* : gf* = *$.

    Now we need to prove 
    $f\circ g \sim \id_S := \prod_{x:S} fgx = x$.
    However, $gx \equiv *$, and $f* \equiv t$, so $fgx \equiv t$.
    Then $p$ is already of this type. I.e., 
    by definition, we have $\ds p:\prod_{x:S} fgx = x$.

    This completes the proof.
\end{proof}

HW: Consider a dependent type, $b:B\yields E(b) : U$ such that
$b:B\yields c(b):\isContr(E(b))$, then 
\[\sum_{b:B}E(b) \simeq B. \]

This generalizes a problem on HW3, where we prove
\[\sum_{b:B} \top \simeq B.\]

\begin{proposition}
    Given $f:A\to B$,
    \[\isEquiv(f) \simeq 
    \prod_{b:B} \isContr\of*{
        \sum_{a:A} fa = b
    }
    \]

    I.e., $f$ being an equivalence is equivalent 
    to all the fibers of $f$ being contractible.
\end{proposition}

\begin{proposition}
    For any type $T$, and $t:T$, the type 
    \[
        \sum_{u:T} t=u
    \]
    is contractible.

    Note that this is an English translation of the type theory 
    statement:
    \[
        T:U,t:T\yields ?:\isContr\of*{
            \sum_{u:T} t=u
        }
    \]

    Topologically (roughly), this is the statement that the 
    universal cover of a path connected space is simply connected.
    Note that we have a point in this space for every path out 
    of $t$.
\end{proposition}

\begin{proof}
    Center of construction,
    $(t,\refl_t)$.
    \[\prod_{s:\sum_{u:T}t=u} \of*{
        (t,\refl_t) = s
    }
    \simeq
    \sum_{p:t=\pi_1s}
    \of*{
        p_*\refl_t = \pi_2s
    }
    \]
    
    HW: prove that we have 
    $\refl_{\refl_t}: \refl_{t,*}\refl_t = \refl_t$
    by showing $\refl_{t,*}\refl_t \equiv \refl_t$.
\end{proof}
    


\end{document}