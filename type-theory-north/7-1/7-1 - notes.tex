\documentclass{article}

\title{Type Theory with Paige North}
\author{Jason Schuchardt}

\usepackage{jragonfyre}

\begin{document}
\maketitle

Type theory is a relatively new branch of math. 
It cannibalizes a lot of other areas of mathematics.
In particular it has connections to logic, set theory,
and functional programming. It is the basis of a lot
of modern functional programming. It also takes a 
lot of inspiration from category theory, and 
its subfield, topos theory. It's also related
to homotopy theory.

The map:
\[
\begin{tikzcd}
\text{Formal Verification} \arrow[rd, no head]                          & \text{Logic} \arrow[rd, no head] \arrow[d, no head]       & \text{Foundations of Mathematics} \arrow[d, no head]                          \\
\text{Functional Programming} \arrow[r, no head]                        & \text{Type Theory} \arrow[r, no head] \arrow[ru, no head] & \text{Set Theory}                                                             \\
\text{Homotopy Theory} \arrow[ru, no head] \arrow[r, no head]           & \text{Category Theory} \arrow[u, no head]                 & \text{Topos Theory} \arrow[lu, no head] \arrow[l, no head] \arrow[u, no head] \\
\text{Practice of Mathematics} \arrow[ru, no head] \arrow[ruu, no head] &                                                           &                                                                              
\end{tikzcd}
\]

\section{Basics}

Basic objects are \emph{types} and \emph{terms}.
Every term belongs to exactly one type. When the term $t$
belongs to a type $T$, then we write $t:T$.

\begin{example}
    In set theory, one writes $5\in\NN$.
    In type theory, one writes $5:\NN$.
\end{example}

In type theory, every term belongs to exactly one type. This
is not the case in set theory.

\begin{example}
    In set theory, one can think of $5$ by itself. It is itself
    a set. One can then say that $5\in \NN$, and $5\in\RR$.

    This is not the case in type theory. In type theory,
    one can only consider a term, like $5$, as being part of 
    a type, $\NN$.

    The terms $5:\NN$ and $5:\RR$ are completely different 
    things. We might be able to compare them in a few classes,
    but they aren't immediately comparable. They are distinct.
\end{example}

History: The first person to come up with a sort of type
theory was Bertrand Russell in 1902. He invented it to solve
Russell's paradox (Is there a set that contains all the sets 
that don't contain themselves). However, 
it was very different from
what we'll be talking about.

\section{A first type theory: The Simply Typed Lambda Calculus}

The simply typed lambda calculus (Church 1940) is the simplest
type theory along the lines of what we're thinking about,
and a model for computation (functional programming).

\begin{definition}
    A simply typed lambda calculus with $\implies$ consists of 
    the following:
    \begin{itemize}
        \item Atomic types $T_1,\ldots,T_n$.
        \item A type $S\implies T$ ($S$ ``implies'' $T$) for 
            every two types $S$ and $T$.
        \item For each type $T$, we have variables 
            \[ x_1^T,\ldots, x_m^T : T, \]
            and for any term $t:T$ and variable $x:S$, we have
            the term $\lambda x.t : S\implies T$.

            We say we are \emph{abstracting $x$ from $t$}.
            The term 
            $t$ might involve $x$ (if it doesn't then the 
            ``function'' we are defining is a constant
            function).
        \item For every term $f:S\implies T$, and every term 
            $s:S$, we have a term $fs : T$.
            We call this term the \emph{application of $f$ to $s$}.
    \end{itemize}
    These are subject to the equations 
    \begin{itemize}
        \item For every term $t:T$, variable $x:S$, term $s:S$,
            \[ (\lambda x.t)s = t[s/x], \]
            where $t[s/x]$ is the term obtained from 
            $t$ by replacing every instance
            of $x$ with $s$.
        \item For each $f:S\implies T$ and variable $x:S$
            which doesn't occur in $f$ (so we don't have variable
            naming conflicts basically). Then we have 
            \[ \lambda x.(fx) = f : S\implies T \]
    \end{itemize}

\end{definition}

Another example of a type is $\NN$. $\NN$ is a type with 
terms $0:\NN$ and $Sn :\NN$, where $n:\NN$ is a term.
$0$ is a \emph{closed term of $\NN$}.

$T_1$ is an atomic type, but $T_1\implies T_2$ is nonatomic.
$(T_1\implies T_2)\implies T_3$ is also nonatomic.

\subsection{Things we can do!}

\begin{example}
For every term $t:T$ not containing all variables of type $S$,
there is a term of $S\implies T$.
We need to supply a variable $x:S$ and our term $t:T$ to get 
such a 
term. For example $\lambda x_j^S.t:S\implies T$, where $x_j^S$
doesn't occur in $t$.

Then \[ (\lambda x_j^S.t) s = t[s/x_j^S] = t. \] 
This is a constant 
function that sends everything in $S$ to our term $t$.
\end{example}

\begin{example}
    For every type $T$, there is a term $T\implies T$.
    Let $x:T$ be a variable. We can build the identity function:
    \[ \lambda x.x : T\implies T. \]

    The first equation tells us that 
    \[ (\lambda x.x)t = x[t/x] = t,
    \]
    which is why we can call it the identity function.

    Question: Do we have a different identity function for each
    variable?

    Answer: We can prove that they are the same. In type theory
    we distinguish between syntax and semantics.
    The formulas $\lambda x.x$ and $\lambda y.y$ are 
    \emph{different} formulas, in a sort of naive sense. However
    their semantics are somehow the same.

    For the proof, we can use the 
    function variable replacement rule (the second equation)
    \[ \lambda x.(fx) = f. \]
    If $x$ and $y$ are variables, then let $f=\lambda y.y$.
    \[ \lambda y.y = f = \lambda x.(fx) 
    = \lambda x.((\lambda y.y)x) = \lambda x.x. \]
\end{example}

This is a basic theory of functions. We have a type for functions,
$S\implies T$, and an identity function, 
$\lambda x.x : T\implies T$.

Note! The only functions we have are ones for which we can write 
down an explicit formula. E.g. $\lambda x.x$, $\lambda x.t$.

This is \emph{not} a theory of \emph{mathematical} 
functions (in the sense of Set Theory), but \emph{computational}
functions!

What is a ``mathematical'' function. Officially, in
set based math, a function $f:X\to Y$ is defined as a graph 
$G\subset X\times Y$. Intuitively, this set is something
like a table, that records for each $x\in X$, the value
$f(x)\in Y$. For example, for the function
\[ \lambda x.x^2, \]
(the function $f(x)=x^2$) from $\RR$ to $\RR$.
The set theoretic representation of this function will be
as the set of 
pairs \[ \set{(0,0),(1,1),(2,4),(\sqrt{2},2),\ldots}\]

Even though the functions we consider in math usually have
nice and finite formulas, a mathematical function doesn't
have to have a formula, and might only be describable by 
such an infinite table.

Neither humans nor computers can work with infinite descriptions,
so if we're interested in studying functions in a computational
setting, we need a finite formula for every function.

In type theory, we have the interpretation
\[ \text{Functions} \longleftrightarrow \text{Programs}. \]
Under this interpretation
\begin{align*}
     \text{Types}& \longleftrightarrow 
\text{Specifications of programs}\\
\text{Terms} &\longleftrightarrow 
\text{Programs fulfilling the specification}\\
 S\implies T &\longleftrightarrow 
\text{Programs which take input $s:S$ and return $t:T$}
\end{align*}


\end{document}