\documentclass{article}


\title{Type Theory with Paige North 7/5}
\author{Jason Schuchardt}

\usepackage{jragonfyre}
\usepackage{turnstile}
\newcommand{\yields}{\vdash}
\newcommand\blank{ \underline{\phantom{$\quad\quad$}} }
\newcommand\TYPE{\text{ \scshape{type}}}
\newcommand\ctxt{\text{ ctxt}}

\theoremstyle{remark}
\newtheorem*{question}{Question}
\newtheorem*{fact}{Fact}
\newtheorem{exercise}{Exercise}

%\theoremstyle{remark}
%\newtheorem{exercise}

\begin{document}

\maketitle

Correction:

Last time we wrote the rule
\[
    \frac{T\TYPE}{x_i^T:T},
    \]
but this should really be 
\[
    \frac{T\TYPE}{x_i^T:T\yields x_i^T:T},
    \]

\section{And Types}

\begin{definition}
    The rules for $\wedge$-types are the following
    \begin{enumerate}
    \item $\wedge$-type formation:
    \[
        \frac{S\TYPE,\quad T\TYPE}{S\wedge T \TYPE},
    \]
    \item term construction \[
        \frac{\Gamma \yields s:S, \quad \Gamma\yields t:T}
        {
            \Gamma \yields (s,t) : S\wedge T
        },
    \]
    \item and term deconstruction
        \[ \frac{\Gamma \yields c : S\wedge T}{
            \Gamma\yields \pi_1(c):S,\quad \Gamma\yields \pi_2(c):T
            }
        \]
    \item compatibility
        \[
            \frac{\Gamma\yields s:S,\quad \Gamma \yields t:T}{
                \Gamma \yields \pi_1(s,t) = s :S,\quad
                \Gamma\yields\pi_2(s,t) = t:T
            }
        \]
    \end{enumerate}
\end{definition}

\begin{example}
    We get functions $\pi_1:S\wedge T \implies S$ and 
    $\pi_2 : S\wedge T \implies T$.

    These are derived in the following way:
    We start with 
    \[c:S\wedge T \yields c : S\wedge T,\]
    by deconstruction this gives
    \[c:S\wedge T \yields \pi_1(c):S, \]
    and then by lambda abstraction, we get
    \[ \yields \lambda c.\pi_1(c) : (S\wedge T)\implies S. \]

    Whenever we get a function from
    \[\frac{\Gamma,x:S\yields y:T}{
        \Gamma \yields \lambda x.y : S\implies T},\]
    we call this \emph{Currying}, and say we \emph{curry} the $x$.
\end{example}

\begin{example}
    We can derive a term of 
    \[ (S\implies T) \wedge S \implies T. \]

    We start with the following things we know:
    \[ c:(S\implies T) \wedge S \yields \pi_1(c) : S\implies T, \]
    and
    \[ c:(S\implies T) \wedge S \yields \pi_2(c) : S. \]
    Then function application gives
    \[c:(S\implies T) \wedge S \yields \pi_1(c)\ \pi_2(c) : T,\]
    so finally, by lambda abstraction, we have 
    \[ \yields \lambda c.(\pi_1(c)\ \pi_2(c))
    : (S\implies T)\wedge S \implies T \]
\end{example}

\section{Philosophy}

There is a logical interpretation of type theory.
\begin{table}
\centering
\begin{tabular}{ccccc}
    Object & & Programatic View & & Logical View
    \\
    Types
    & $\longleftrightarrow$
    &
    Program Spaces
    & $\longleftrightarrow$
    &
    Propositions
    \\
    Terms
    & $\longleftrightarrow$
    &
    Programs
    & $\longleftrightarrow$
    &
    Proofs
    \\
    Terms in context
    & $\longleftrightarrow$
    &
    Programs that take input
    & $\longleftrightarrow$
    &
    Proofs that take hypotheses
    \\
\end{tabular}
\caption{Interpretations of type theory}
\end{table}

For example, we might have the type 
$12$ is composition, a proof might be $12 = 3\cdot 4$.
Another proposition/type would be $T$ has a term, a proof might 
require hypotheses/context, and we could use the context to 
produce a term of $T$.

\begin{example}
    $f:S\implies T$ is a function that turns a proof of $S$ into
    a proof of $T$.

    For example, if $S=\text{``$12$ is composite.''}$,
    and $T=\text{``$24$ is composite.''}$.
\end{example}

\begin{example}
    A term of the type $c:S\wedge T$ represents a proof of 
    $S$ and $T$, since from $c$ we can obtain proofs
    $\pi_1(c):S$ and $\pi_2(c):T$. 

    As a silly example, we could say $S=\text{``12 is composite''}$,
    and $T=\text{``4 is composite''}$.
\end{example}

In the logical interpretation, the term we just constructed of
type $(S\implies T)\wedge S \implies T$, tells us that 
if we know $P\implies Q$, and $P$, then $Q$ is true. This 
rule is called \emph{modus ponens} in propositional logic.

This logic/program correspondence is called the Curry-Howard
correspondence, and dates from 58-69. It is also called 
the Proofs-as-programs paradigm.

Usually in mathematics, proofs are often thought of as
metamathematical. In type theory, proofs are actual mathematical 
objects. Thus we say type theory is \emph{proof relevant}.
It is often important that there is more than one proof.

Note that in type theory, proofs are always constructive.
This is because we are always constructing a term in a type.
This is constructive mathematics, and we don't use the law of 
excluded middle.

\section{Disjunction: $\vee$-types}

\begin{definition}
    Rules for $\vee$-types.

    \begin{enumerate}
        \item type construction:
            \[ \frac{S\TYPE,\quad T\TYPE}{S\vee T\TYPE}\]
        \item term construction:
            \[\frac{\Gamma\yields s:S}{\Gamma\yields i_1(s):S\vee T}\]
            \[\frac{\Gamma\yields t:T}{\Gamma\yields i_2(t):S\vee T}\]
        \item term deconstruction:
            \[ \frac{\Gamma,x:S\yields r:R,\quad 
            \Gamma, y:T\yields r':R}{
                \Gamma, z:S\vee T \yields j_{r,r'}(z) :R
            }
            \]
        \item compatibility
            \[ \Gamma,x:S\yields j_{r,r'}(i_1(x))=r:R \]
            \[ \Gamma,y:S\yields j_{r,r'}(i_2(y))=r':R \]
    \end{enumerate}
\end{definition}

To construct a term of $S\vee T$, we can either supply an $s:S$,
or a $t:T$. In the logical interpretation, we can 
say that to prove $S\vee T$, it suffices to prove $S$
or $T$.

To construct a term of $R$ from a term of $S\vee T$, 
we can supply functions $f:S\implies R$, and $g:T\implies R$.
In the logical interpretation, this says that if we want to 
prove $R$ using $S\vee T$, we can prove $S\implies R$ and 
$S\implies T$.

\begin{example}
    From these rules, we get some functions:
    \begin{enumerate}
        \item $i_1=\lambda s.i_1(s) : S\implies S\vee T$,
        \item $i_2=\lambda t.i_2(t) : T\implies S\vee T$,
    \end{enumerate}

    Now we want to prove 
    \[R\wedge (S\vee T) \implies (R\wedge S)\vee (R\wedge T).\]
    We begin with 
    \[
    \frac{
        x:R,y:S\yields i_1(x,y) :(R\wedge S)\vee (R\wedge T),
        \quad 
        x:R,z:T\yields i_2(x,z) : (R\wedge S)\vee (R\wedge T)
    }{
        x:R,w:S\vee T \yields j_{i_1,i_2}(x,w) : 
        (R\wedge S) \vee (R\wedge T)
    }
    \]
    Then we can conclude 
    \[ 
        v:R\wedge (S\vee T) \yields j_{i_1,i_2}(\pi_1v,\pi_2v):
        (R\wedge S) \vee (R\wedge T).
    \]
    Finally, we lambda abstract to get
    \[
        \yields 
        \lambda v.j_{i_1,i_2}(\pi_1v,\pi_2v):
        R\wedge (S\vee T)\implies (R\wedge S) \vee (R\wedge T).
    \]

    Question: In what sense is the string unambiguous? Can 
    we work out things from the type?
    Answer: We're abbreviating slightly, but we can work 
    out the types of the important things from the type 
    of the whole expression.

    Note from note taker, we're eliding a rather important 
    piece from this proof, where we construct a function
    $i_r:S\implies R\wedge S$.
\end{example}

\section{Bottom and Top}
\begin{definition}
    We have the rules for true or top: $\top$,
    and false or bottom: $\bot$
    \begin{enumerate}
        \item \[ \frac{}{\top\TYPE}, \quad
        \frac{}{* : \top}\]
        \item \[\frac{}{\bot\TYPE},\quad
        \frac{\Gamma \yields f:\bot,\quad 
        S\TYPE}{
            \Gamma \yields j_f:S
        }\]
    \end{enumerate}
\end{definition}

\begin{exercise}
    Prove $S\implies \top$, $\bot \implies S$.
\end{exercise}

\begin{definition}
    Define $\lnot S$ as $S\implies \bot$.
    We cannot construct a term of 
    \[S\vee \lnot S\]
    for every $S$.
\end{definition}

This corresponds to unprovable statements in logic.
This fact is related to G\"odels incompleteness theorem.

A term of $S\vee \lnot S$ is called the 
\emph{law of excluded middle}.

We can't prove the law of excluded middle in type theory, 
and we can't prove its negation, so the law of excluded middle
is \emph{independent} of the theory.

We say a statment is \emph{independent} of axioms/a proof system
if when you add that statement, you cannot prove false, 
and when you add the negation of that statement, you still
cannot prove false.
The existence of independent statements means that the 
theory is incomplete.

We \emph{model} theory in different categories. Since 
a model is more concrete, we often have statements that 
are true in the model that are not true in the theory.

When we construct models, we are looking for more complete
systems.

We can construct a model in the category of sets. Types are 
sets, terms are elements, implies is the set of functions,
$\wedge=\times$, $\vee = \sqcup$, $\top=\set{*}$,
$\bot = \nullset$. Then in $\mathbf{Set}$,
\[S\vee \lnot S = S\sqcup (S\to \nullset).\]
Can we construct a term of this type? (I.e., an element 
of this set.)
If $S=\nullset$, then there is an element of 
$S\to \nullset$, so $S\sqcup (S\to\nullset)$ is nonempty,
and if $S$ is nonempty, then $S\sqcup 
(S\to\nullset)$ is nonempty. Thus in either case, the 
law of excluded middle holds for $\mathbf{Set}$.

The law of excluded middle fails for presheaves on the 
arrow category. I.e., the category
\[[2,\mathbf{Set}].\]
The objects are triples $(A,B,f)$, with $A,B$, sets and 
$f:A\to B$ a function. The arrows in this category
are pairs $(g,h):(A,B,e)\to (C,D,f)$,
of functions $g:A\to C$, $h:B\to D$ such that 
\[
    \begin{tikzcd}
        A\ar[r,"e"]\ar[d,"g"] & B\ar[d,"h"] \\ % TODO
        C\ar[r,"f"] & D 
    \end{tikzcd}
    \]
commutes.

Then $\bot = (\nullset,\nullset,\id)$,
$\top = (*,*,\id)$. Then we can show that for some objects $S$
in this category,
$S\vee \lnot S$ is `empty' generally.
Terms of an object $S$ here are elements of $\Hom(\top,S)$.

\end{document}